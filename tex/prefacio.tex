\chapter*{PREFACIO}
\addcontentsline{toc}{chapter}{PREFACIO}

Se recalca que esta obra son notas del curso de Cálculo III, del periodo de 2024/2. Es probable que exista algún error ortográfico o matemático, por lo que se recomienda consultar la bibliografía proporcionada al final de este trabajo. La presente obra consta de todo el contenido que adjudicó el Lic. Andrés Sabino Díaz Castro, quien imparte clases en la Escuela Superior de Física y Matemáticas (ESFM) en el Instituto Politécnico Nacional (IPN). Hice el mayor esfuerzo en estructurar el libro a base de proposiciones, ejercicios, notas, etc.; y se agregaron y modificaron algunas definiciones, a fin de lograr una mayor comprensión sin llegar a tener alguna ambigüedad.


Cada capítulo contiene varios ejercicios y demostraciones. Estos son de diversos tipos y pueden ayudar a tener una mejor comprensión del tema. Cabe mencionar que este libro fue mecanografiado por Marco Antonio Molina Mendoza, estudiante de la Escuela Superior de Física y Matemáticas. Si lo deseas o lo requieres, puedes imprimir esta obra, eres libre de hacerlo y no necesitas autorización. Cualquier mención a esta obra es apreciada, pero no requerida. Se sigue trabajando para corregir errores y/o mejorar este trabajo, por lo que está en constante cambio. Se recomienda entrar en  \href{https://linktr.ee/biblioteca_esfm}{\textbf{Biblioteca ESFM}} para encontrar la versión más actualizada.\marginElement{
\begin{center}
    \begin{tikzpicture}
        \node[fill=white] at (0,0) {\hypersetup{hidelinks}\qrcode[hyperlink, height=0.75\linewidth]{https://linktr.ee/biblioteca_esfm}};
    \end{tikzpicture}
\end{center}
}

Algunos cambios han sido resultado de los comentarios de mis amigos y estudiantes de la Escuela Superior de Física y Matemáticas. Estas son algunas de las muchas mejoras que se han incorporado en esta edición.
\begin{itemize}
    \item Aspecto estético mejorado: He trabajado arduamente en mejorar el aspecto visual del libro. Se han incorporado nuevas fuentes y estilos para que sea más fácil de leer y asimilar los conceptos.
    \item Errores ortográficos y matemáticos corregidos: He realizado una revisión minuciosa para garantizar la precisión en todos los aspectos del contenido. Los errores ortográficos y matemáticos se han corregido para ofrecer una experiencia de aprendizaje fluida y confiable.
    \item Imágenes actualizadas: He renovado todas las imágenes y gráficos del libro para asegurarme de que sean claros, informativos y visualmente atractivos. Las nuevas ilustraciones ayudarán a comprender mejor los conceptos y aplicaciones del álgebra lineal en el mundo real.
    \item Ejemplos adicionales y ejercicios prácticos: He añadido más ejemplos paso a paso y problemas resueltos para reforzar el entendimiento de los temas discutidos.
    %\item Recuadros de resumen y notas destacadas: He agregado recuadros de resumen al final de cada sección para recapitular los puntos clave. Además, he incluido notas destacadas para enfatizar conceptos importantes y recordatorios útiles a lo largo del libro.
\end{itemize}
Quiero destacar que este libro, disponible en formato PDF, ofrece una funcionalidad adicional a través de tres hipervínculos de gran utilidad.
\begin{itemize}
    \item Números de páginas: Se ubican en la parte superior derecha e izquierda, dependiendo de si es una página par o impar. Al hacer clic, te redirigirán al índice del libro.
    \item Nombres de secciones en el encabezado: Se localizan en la parte superior derecha de cada página impar. Al hacer clic, te llevarán al inicio de la respectiva sección
    \item Nombres de capítulos en el encabezado: Se encuentran en la parte superior izquierda de cada página par. Al hacer clic, te dirigirán al inicio del correspondiente capítulo.
\end{itemize}

En la elaboración de este libro, he optado por una presentación que refleje la esencia misma de la disciplina: precisión, claridad y elegancia matemática. Con este propósito en mente, he desarrollado una plantilla en LaTeX que resalta la pureza del contenido, utilizando únicamente el contraste entre el blanco y el negro para enfocar la atención en los conceptos fundamentales que exploraremos a lo largo de estas páginas.

La elección de un diseño en blanco y negro no es solo estética, sino una decisión que busca facilitar la comprensión del material para los lectores. Al eliminar cualquier distracción visual que pueda surgir del uso de colores llamativos, me he enfocado en ofrecer una experiencia de lectura que permita una inmersión total en el mundo del Cálculo.

Las ecuaciones, teoremas y demostraciones, que constituyen el núcleo de este texto, se presentan de manera clara y legible, sin interferencias visuales. La tipografía se ha seleccionado con cuidado para garantizar una lectura fluida y agradable, mientras que las figuras y gráficos han sido diseñados con precisión para complementar y reforzar los conceptos expuestos en el texto.

Debo mencionar que las imágenes que se presentan en este libro están hechas con \texttt{TikZ}, una herramienta para la producción de gráficos vectoriales. Dichas imágenes son de elaboración propia y son de dominio público.

Entiendo la importancia de la legibilidad tanto en la versión impresa como en la digital, por lo que he dedicado especial atención a la adaptabilidad de mi plantilla. Ya sea en un libro físico o en un documento electrónico, los lectores encontrarán una presentación coherente y accesible que facilitará su inmersión en los temas tratados.

Con una presentación clara y concisa de los temas, busco fomentar la apreciación de la belleza y utilidad de esta rama de las matemáticas. Se aspira a que los lectores adquieran las habilidades y conocimientos necesarios para resolver problemas complejos y se sientan motivados a explorar aplicaciones más avanzadas del cálculo en su futuro académico y profesional.\vfill

\begin{flushright}
    Marco Antonio Molina Mendoza\\ 
    México, \today\\ 
\end{flushright}